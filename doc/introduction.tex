\section{Introduction}

Currently most user work with the \ac{TOF} data analysis tool
\citep{Muller2013a} provided by Ionicon.  Here another work flow is
developed based on a set of principles.

\subsection{Principles}

\paragraph{Data}

Assuring the integrity of observed data is of paramount importance.  One
core point is that any file containing observation data must not be
altered.  The integrity of such files can then also be confirmed using
checksum and even cryptographic signatures to guarantee authenticity.


\paragraph{batch processing}

For larger measurement campaigns a reproducible work flow is essential.
This means that every measurement must be processed exactly the same
way.  This sounds like a non-issue, but in practice is difficult to
achieve.  Usually the course of any campaign issues appear which
require adjusting the evaluation.

Doing this manually one file at a time is not only laborious, but also a
great risk of error.  A tool that will allow for reevaluation of a large
set of measurements with spelled out parameter can help in this
process.


\paragraph{modularity}

Making the evaluation modular has two great benefits.  For one users can
replace any module with one that suits their need; especially as modules
can be created directly by the user.  Second the intermediate
results between modules may be stored for reuse.  The latter requires
dependency tracking, but we will show how this can easily be achieved.

\paragraph{Mass calibration}

The state of mass calibration provided leaves some room for improvement.
For one it uses the buf as smallest unit, although this averages over a
number of scans.  The next issue is that peaks are actually fit to a
curve; This does not seem reasonable for noisy peaks.  Third when
plotted the coefficients of the calibration function do not follow a
smooth curve, my personal interpretation is that here noise in the
estimation of these coefficients manifests.  Lastly I am worried that
the whole calibration hangs on singular masses.  

\subsection{outline}

The mass calibration actually performs two tasks.  One it corrects for
drifts in the flight time; and two it translates flight time to mass.

It is difficult to correct for drifts, but recently a family of
corrections were introduced under the name of \emph{time warping}.  For
\ac{TOF} spectra the parametric time warp \citep{Eilers2004a} is appropriate.

Once the drift has been eliminated mass calibration can be performed on
recalculated sum spectra, which contain peaks that may be absent from
individual spectra.

For integration we provide modules to integrate predefined masses as
well as automatic detection of peaks in the sum spectrum and calculating
their traces. 

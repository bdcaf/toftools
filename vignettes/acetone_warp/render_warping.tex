\section{Warping}


\begin{kframe}


{\ttfamily\noindent\bfseries\color{errorcolor}{\#\# Error in stored\_mass\_cal.tof\_h5(H5Fopen(data\_files[[1]])): konnte Funktion "{}stored\_mass\_cal.tof\_h5"{} nicht finden}}\end{kframe}

\begin{figure}
\includegraphics[width=\maxwidth]{render_warping/avspec-1} \caption[Average spectrum over all samples]{Average spectrum over all samples}\label{fig:avspec}
\end{figure}






As we see in \fref{fig:warp1} the best overlap is at \num{0} shift.  So warp correction is not necessary.

\begin{figure}
\includegraphics[width=\maxwidth]{render_warping/wide_warp1-1} \caption[Match factor of first spectrum under shift, showing wider range]{Match factor of first spectrum under shift, showing wider range.}\label{fig:wide_warp1}
\end{figure}



\begin{figure}
\includegraphics[width=\maxwidth]{render_warping/warp1-1} \caption[Match factor of first spectrum under shift]{Match factor of first spectrum under shift.}\label{fig:warp1}
\end{figure}



\begin{figure}
\includegraphics[width=\maxwidth]{render_warping/warp6-1} \caption[Match factor of sixth spectrum under shift]{Match factor of sixth spectrum under shift.}\label{fig:warp6}
\end{figure}





\begin{figure}
\includegraphics[width=\maxwidth]{render_warping/showwarps-1} \caption[actual warps when compared to sum spectrum]{actual warps when compared to sum spectrum.}\label{fig:showwarps}
\end{figure}



The required shifts are tiny -- probably just related with the saturated peaks.

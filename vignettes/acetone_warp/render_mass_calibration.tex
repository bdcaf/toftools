\section{Mass calibration}

As good alignment has been assured beforehand now mass calibration can be performed on the sum spectrum.



\begin{kframe}


{\ttfamily\noindent\bfseries\color{errorcolor}{\#\# Error in data\_files[[1]]: Indizierung außerhalb der Grenzen}}

{\ttfamily\noindent\bfseries\color{errorcolor}{\#\# Error in min(.): ungültiger 'type' (list) des Argumentes}}

{\ttfamily\noindent\bfseries\color{errorcolor}{\#\# Error in min(.): ungültiger 'type' (list) des Argumentes}}\end{kframe}

\begin{kframe}


{\ttfamily\noindent\bfseries\color{errorcolor}{\#\# Error in eval(expr, envir, enclos): Objekt 'mcfun' nicht gefunden}}

{\ttfamily\noindent\bfseries\color{errorcolor}{\#\# Error in eval(expr, envir, enclos): Objekt 'agg\_spec' nicht gefunden}}

{\ttfamily\noindent\bfseries\color{errorcolor}{\#\# Error in plot(mass\_ax0, sig, type = "{}l"{}, col = "{}red"{}, ylab = "{}signal"{}, : Objekt 'mass\_ax0' nicht gefunden}}\end{kframe}




At least three ions are required to perform proper mass calibration.  The
selected ions are \ce{H3O+}(mz=\num{21.022}),
acetone(mz=\num{59.049}), D6
acetone(mz=\num{65.087}),
isoprene(mz=\num{69.07}),
ethanol(mz=\num{47.049}) and
acetaldehyde(mz=\num{45.033}).


\Fref{fig:plotuncorrected} shows there is only a minor deviation between the old mass calibration and observed signals.

\begin{kframe}


{\ttfamily\noindent\bfseries\color{errorcolor}{\#\# Error in FUN(X[[i]], ...): Objekt 'mass\_ax0' nicht gefunden}}\end{kframe}

\begin{kframe}


{\ttfamily\noindent\bfseries\color{errorcolor}{\#\# Error in FUN(X[[i]], ...): Objekt 'mcfun' nicht gefunden}}

{\ttfamily\noindent\bfseries\color{errorcolor}{\#\# Error in is.data.frame(data): Objekt 'max\_locations' nicht gefunden}}\end{kframe}

In \fref{fig:masscalibs} the resulting mass calibration curve is shown.
On this scale the difference to the stored one is not recognisable.  The
stored curve is plotted in blue but completely overlapped be the newer fit.
However the difference can be seen when looking at the difference of points
on the calibrated lines to their theoretical values shown in \fref{fig:mcdiffs}.

\begin{kframe}


{\ttfamily\noindent\bfseries\color{errorcolor}{\#\# Error in predict(fitm, newdata = data.frame(mass = plot\_masses)): Objekt 'fitm' nicht gefunden}}

{\ttfamily\noindent\bfseries\color{errorcolor}{\#\# Error in eval(expr, envir, enclos): Objekt 'mcfun' nicht gefunden}}

{\ttfamily\noindent\bfseries\color{errorcolor}{\#\# Error in with(max\_locations, plot(mass, index)): Objekt 'max\_locations' nicht gefunden}}

{\ttfamily\noindent\bfseries\color{errorcolor}{\#\# Error in xy.coords(x, y): Objekt 'orig\_cal\_indexes' nicht gefunden}}

{\ttfamily\noindent\bfseries\color{errorcolor}{\#\# Error in xy.coords(x, y): Objekt 'plot\_inds' nicht gefunden}}\end{kframe}

\begin{kframe}


{\ttfamily\noindent\bfseries\color{errorcolor}{\#\# Error in predict(fitm, newdata = data.frame(mass = unlist(masses))): Objekt 'fitm' nicht gefunden}}

{\ttfamily\noindent\bfseries\color{errorcolor}{\#\# Error in eval(expr, envir, enclos): Objekt 'mcfun' nicht gefunden}}

{\ttfamily\noindent\bfseries\color{errorcolor}{\#\# Error in plot(masses, pio - max\_locations\$index, col = "{}blue"{}, ylab = "{}difference"{}): Objekt 'pio' nicht gefunden}}

{\ttfamily\noindent\bfseries\color{errorcolor}{\#\# Error in xy.coords(x, y): Objekt 'pi2' nicht gefunden}}\end{kframe}

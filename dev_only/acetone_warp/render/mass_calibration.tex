\section{Mass calibration}

As good alignment has been assured beforehand now mass calibration can be performed on the sum spectrum.





\begin{figure}
\includegraphics[width=\maxwidth]{render/mass_calibration/avspecmass-1} \caption[Average spectrum over all samples]{Average spectrum over all samples}\label{fig:avspecmass}
\end{figure}






At least three ions are required to perform proper mass calibration.  The
selected ions are \ce{H3O+}(mz=\num{21.022}),
acetone(mz=\num{59.049}), D6
acetone(mz=\num{65.087}),
isoprene(mz=\num{69.07}),
ethanol(mz=\num{47.049}) and
acetaldehyde(mz=\num{45.033}).


\Fref{fig:plotuncorrected} shows there is only a minor deviation between the old mass calibration and observed signals.

\begin{figure}
\includegraphics[width=\maxwidth]{render/mass_calibration/plotuncorrected-1} \caption[uncorrected masses]{uncorrected masses}\label{fig:plotuncorrected}
\end{figure}





In \fref{fig:masscalibs} the resulting mass calibration curve is shown.
On this scale the difference to the stored one is not recognisable.  The
stored curve is plotted in blue but completely overlapped be the newer fit.
However the difference can be seen when looking at the difference of points
on the calibrated lines to their theoretical values shown in \fref{fig:mcdiffs}.

\begin{figure}
\includegraphics[width=\maxwidth]{render/mass_calibration/masscalibs-1} \caption[Mass calibration curve through predetermined masses]{Mass calibration curve through predetermined masses.  Stored curve in red, new fit in blue -- but they almost completely overlap.}\label{fig:masscalibs}
\end{figure}



\begin{figure}
\includegraphics[width=\maxwidth]{render/mass_calibration/mcdiffs-1} \caption[differences of mass calibration to observation]{differences of mass calibration to observation. New curve is red, stored in blue}\label{fig:mcdiffs}
\end{figure}



\section{Integration of peaks}

As good alignment has been assured beforehand now mass calibration can be performed on the sum spectrum.







The example peaks to be integrated are shown in \fref{tab:integrpeaks}

% latex table generated in R 3.4.3 by xtable 1.8-2 package
% Fri Dec  8 13:15:01 2017
\begin{table}[ht]
\centering
\begin{tabular}{rlrrrr}
  \hline
 & name & mass & wid & low & high \\ 
  \hline
1 & h3o & 21.02 & 0.02 & 21.01 & 21.04 \\ 
  2 & acetone & 59.05 & 0.03 & 59.02 & 59.07 \\ 
  3 & d6acetone & 65.09 & 0.03 & 65.06 & 65.11 \\ 
  4 & isoprene & 69.07 & 0.03 & 69.04 & 69.10 \\ 
   \hline
\end{tabular}
\caption{parameter for peak integration} 
\label{tab:integrpeaks}
\end{table}






To show how integration works first consider a single file.
And d6acetone was integrated.

\begin{figure}
\includegraphics[width=\maxwidth]{render/peak_integration/plotSpec65-1} \caption[peak spectrum of D6 in example measurment]{peak spectrum of D6 in example measurment}\label{fig:plotSpec65}
\end{figure}



\begin{kframe}


{\ttfamily\noindent\bfseries\color{errorcolor}{\#\# Error in plot.window(...): endliche 'ylim' Werte nötig}}\end{kframe}\begin{figure}
\includegraphics[width=\maxwidth]{render/peak_integration/plotProf65-1} \caption[slice of time profile of D6 in example measurment]{slice of time profile of D6 in example measurment}\label{fig:plotProf65}
\end{figure}


